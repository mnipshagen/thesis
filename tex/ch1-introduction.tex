% !TeX root = ./main.tex

\section{Introduction}
Accurately supporting students and preventing dropouts is one of the more daunting tasks universities face in the current age. Students have different reasons for dropping out of their program, and not all of them are preventable nor should they be prevented. If a student finds a better fit elsewhere or has a grant opportunity to follow they should not be held back. However, those that drop out because they struggle with the subjects or extracurricular burdens, should be actively supported. For a personalized intervention to be effectively planned and executed, struggling students need to be identified early on. In order to identify students that struggle to complete their program, those features must be identified which constitute to a successful study episode.\\
Universities and institutions collect a wide range of data on students' background, progress and success. To analyze these large amounts of data, techniques from fields like data mining, artificial intelligence and machine learning are useful. Bringing data mining techniques into the educational setting has been a growing scientific field\cite{Romero.2013}. There has been work on high school students to predict and support their study success \cite{Kaplan.1997, Tamhane.2014, Lakkaraju.2015} and faculty specific research \cite{Kovacic.2012, Nikolovski.April2015}. Most of this work is done on small datasets (< 20 variables), or employ only simple techniques for prediction, like logistic regression. Some papers employed decision trees to evaluate and explain the students' success rate. Only one paper was found that used machine learning in a longitudinal study \cite{Tamhane.2014}. The data used in those papers was created for the study itself and did not seem to be publicly available.\\
After features are identified, there is the important and difficult task to support students in achieving their goals. There is ongoing research on this matter. One of those projects is the SIDDATA project \cite{Osada.2019}. SIDDATA aims to provide a data-driven assistant to help students formulate, work towards and achieve individual goals. One main aspect of this assistant is to integrate previously unconnected data sources and provide a combined overview. This can include but is not limited to data from campus management systems, open educational resources and workshop and courses offered at universities.

To identify relevant features, there needs to be representative data to analyze and derive the features from. The mentioned papers all were not located in western or central Europe and their dataset were not publicly available. Hence, there was no reference dataset directly available which is representative for the students in Germany. Websites like \href{https://www.re3data.org/}{re3data.org}, \href{https://nces.ed.gov/datalab/index.aspx}{nces.ed.gov/datalab} and \href{http://educationaldatamining.org/}{educationaldatamining.org} supply a limited selection of datasets concerning education. All of these did not satisfy the crieteria of being longitudinal, large scale and representative. However, the "Leibniz-Institut für Bildungsverläufe" (LIfBi) provides the "National Education Panel Study" (NEPS) dataset\cite{Weinert.}. It features several cohorts (from newborns over kindergarten and school to first-year students and young adults), each a longitudinal study following the participants over the span of several years. The quantitative and qualitative data compiled is exhaustive and with several thousand participants, it was the largest dataset to be found.\\

In line with the approach of the SIDDATA project, this thesis aims to provide a data-driven approach to identify relevant features for student success. This work outlines the process of surveying and preprocessing the NEPS dataset and provides insight into the structure and flaws of the convoluted dataset. Several machine learning algorithms are applied to the preprocessed data and the results analyzed and compared. When working with machine learning, it is important to keep in mind what one is predicting from which source and what this implies. This thesis is looking to predict the successful completion of a degree program. This is predicted from collected background information and episodical contextual information during the study episode. There is no information on skills gained, the quality of the study or the fit of the student for the program entailed. There is a plethora of positive reasons to not finish a study program, but this cannot be evaluated as there is no information on a student's reasoning for dropping out; if it was, it has been redacted. Finally, the important features of those algorithms are read out and discussed. The discussion also contains a more general part on the ethics of employing data mining and AI in education.

All code used for in this thesis can be found on the attached medium, the readme is attached in the appendix \ref{apx:readme}, or on \url{https://github.com/mnipshagen/thesis/}.