% !TeX root = ./main.tex

\section{Conclusion}
In this thesis, the possibility of using the NEPS dataset for a machine learning task to predict success of studies was explored. The surveying of the data and inferred knowledge was presented as well as remarks on improving that procedure. This included insight on target files, relevant to the work on bachelor students, and organization of the files itself. Furthermore, the preprocessing of the data was detailed, in which relevant parts of the NEPS data were identified, and a way of aggregating and streamlining the data into a machine learning compliant format was shown.\\
The preprocessed data was then fed into support vector machines, random forest classifiers, feed forward networks and decision trees. A gridparameter search was employed to evaluate different hyperparameter configurations for each model. The results from the grid search have been munally reviewed and an optimal configuration for each classifier was chosen. and the results analyzed. Although the initial results were troubling due to oversights during preprocessing, measures were introduced to counteract those mishaps. These measures consisted of several corrections and reevaluations.\\
The final results showed that a random forest classifier was the most successful, reaching 80-85\% in score metrics precision, recall, f1 and balanced accuracy. The SVM performed worse probably due to the fact that the data was handled as nominal in all cases, which increased the dimensionality of the feature space by a factor of 16.8.\\
The importance of features of the successful RFC were analyzed and discussed. Most interestingly was that the most important features were a mixture of qualitative and quantitative data. This could have implications for the focus of which data about students is analyzed for predictions made.\\
Lastly, the ethical implications of this work in the SIDDATA project as well as similar contexts were briefly examined, urging to value the students opinion when defining success metrics and respecting the privacy of students when collecting and analyzing data in higher educational institutions.