% !TeX root = ./main.tex

\section*{Abstract}
Students drop out of their program for different reasons. They might have noticed that their current program is not a good fit and look for a better alternative, or they might have had an opportunity for something they considered more valuable than finishing their degree. More problematic are students that drop out due to being overwhelmed, not fulfilling expectations or struggling with extracurricular burdens. Since dropping out of a program often incurs a high financial as well as social cost, it is important to provide personalized interventional support to those students. Using data mining techniques, the large amounts of data collected by universities and institutions can be analyzed. However, most work done in this field has been done on small datasets; additionally some have used only one specific algorithm to deduce a result. The "Leibniz-Institut für Bildungsverläufe" (LIfBi) supplies the expansive "National Education Panel Study" (NEPS) dataset, a longitudinal study with an encompassing breadth of variables recorded. This thesis shows the process of working with the NEPS dataset to create a dataset fit to be used with machine learning. It further employs and compares several machine learning models, whose weights are then analyzed to work out important features for study success.